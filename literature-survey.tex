\chapter{Literature Survey}

\section{Software and Ethics}
Ethics are about right and wrong. But ``right'' and ``wrong'' may not mean exactly the same for everyone. But there are some universal moral imperatives\textsuperscript{\cite{meyer}}:
\begin{itemize}
\item Not causing unjustified loss of human life.
\item Not damaging someone else's reputation through misrepresentation.
\item Not acquiring someone else's legitimate property against his will.
\item Justly remunerating someone else's services (related to preceding one).
\end{itemize}
For various reasons, some people justify making and using unauthorized copies of software. They may not understand the implications of their actions. Unauthorized copying of software can deprive developers of a fair return for their work.

Respect for intellectual labour and creativity is vital to academic discourse and enterprise. This principle applies to works of all authors and publishers in all media. It encompasses respect for the right to acknowledgement and the right to control over the form, manner, and terms of publication and distribution. Because electronic information is volatile and easily reproduced, respect for the work and personal expression of others is especially critical in computer environments.

\section{What the Law says?}
The following is an excerpt from the website of NASSCOM\textsuperscript{\cite{iprindia}} which states the situation of software related law in India (important points of concern have been bold-faced):
\begin{quotation}
In India, the Intellectual Property Rights (IPR) of computer software is covered under the Copyright Law. Accordingly, the copyright of computer software is protected under the provisions of Indian Copyright Act 1957. Major changes to Indian Copyright Law were introduced in 1994 and came into effect from 10 May 1995. These changes or amendments made the Indian Copyright law, one of the toughest in the world.

The amendments to the Copyright Act introduced in June 1994 were in themselves, a landmark in the India's copyright arena. For the first time in India, the Copyright Law clearly explained:
\begin{itemize}
\item the rights of a copyright holder
\item position on rentals of software
\item the rights of the user to make backup copies
\item and most importantly \textbf{the amendments imposed heavy punishment and fines for infringement of copyright of software.}
\end{itemize}

Because most software is easy to duplicate and the copy is usually as good as original, the Copyright Act was greatly in demand. According to this Act, the infringer can be tried under both civil and criminal law.

\textbf{According to section 16 of this Act, it is illegal to make or distribute copies of copyrighted software without proper or specific authorisation.} The only exception is provided by section 52 of the Act, which allows a backup copy purely as a temporary protection against loss, distribution or damage to the original copy. The 1994 amendment to the Copyright Act also prohibits the sale or hiring, or any offer for sale or hire of any copy of the computer program without specific authorisation of the Copyright holder. 
\end{quotation}
Clearly we need to be aware of the prevailing laws before copying, or distributing copies of software. Although individuals can not be held for using illegal copies of software at home but copying or using illegally copied software at work puts the entire company at risk for copyright infringement.

When you purchase your software, you probably buy only the permission, or ``license'', to use it, not to share it. If your software comes with a license agreement, read it carefully before you use the software. Most licenses do not permit you to run your software on two or more computers simultaneously, or to make copies for your friends, your family or your office mates. It is not illegal, however, to loan your original software temporarily to a friend when you are not using it yourself, as long as neither of you makes a duplicate copy, and provided that there is no contrary provision in the license agreement.

\section{Alternatives to Explore}
\subsection{Genuine/Licensed Proprietary Software}
If you can afford to buy your favorite software, do it. But most of the time, software is expensive. There are options such as bulk-purchased software but only applicable to universities and companies. Software available through institutional site licenses or bulk purchases is subject to copyright and license restrictions and you may not make or distribute copies without authorization\textsuperscript{\cite{university}}.
\subsection{Shareware and Trial Software}
Shareware is copyrighted software that the developer encourages you to copy and distribute to others. This permission is explicitly stated in the documentation or displayed on the computer screen. The developer of shareware generally asks for a small donation or registration fee if you like the software and plan to us it. By registering, you may receive further documentation, updates and enhancements. You are also supporting future software development.

Trial Software is usually limited for use within some period of time and after passing of that time, the user is required to either purchase the program or remove it from her system. It is similar to taking a taste of a sweet before buying it.

\subsection{Free or Open Source Software}
Free and open source software (FOSS) or free/libre/open source software (FLOSS) is software that is liberally licensed to grant the right of users to use, study, change, and improve its design through the availability of its source code.

In the context of free and open source software, \emph{free} refers to the freedom to copy and re-use the software, rather than to the price of the software.

\section{Licensing your work\textsuperscript{\cite{license}}}
If you are the copyright owner of a work (and you likely will be if you created the work), such as aprogram, an article, a blog post, a photograph, or a video, you can authorize others to use it. You can do this by transferring to the person who wants to use your work any or all of your rights as a copyright owner. The following are descriptions of some popular free licenses which allow you to give others rights to use your work\textsuperscript{\cite{lic-list}}:
\subsection{Software Licenses}
\begin{itemize}
\item GNU General Public License (GPL)

Nobody should be restricted by the software they use. There are four freedoms that every user should have:
	\begin{itemize}
    \item the freedom to use the software for any purpose,
    \item the freedom to change the software to suit your needs,
    \item the freedom to share the software with your friends and neighbors, and
    \item the freedom to share the changes you make.
	\end{itemize}
When a program offers users all of these freedoms, it is called free software.

Developers who write software can release it under the terms of the GNU GPL. When they do, it will be free software and stay free software, no matter who changes or distributes the program. It is called \emph{copyleft}\footnote{Copyleft is a play on the word \emph{copyright} to describe the practice of using copyright law to offer the right to distribute copies and modified versions of a work and requiring that the same rights be preserved in modified versions of the work. In other words, copyleft is a general method for making a program (or other work) free, and requiring all modified and extended versions of the program to be free as well.\textsuperscript{\cite{copyleft}}}: the software is copyrighted, but instead of using those rights to restrict users like proprietary software does, they are used to ensure that every user has freedom.

This latest version of the GNU GPL is version 3 commonly abbreviated as GPLv3: a free software license, and a copyleft license. It is recommended for most software packages.

\item GNU Lesser General Public License (LGPL)

LGPL is a free software license, but not a strong copyleft license, because it permits linking with non-free modules. It is compatible with GPLv3. It is recommended for special circumstances only.

\item GNU Affero General Public License (AGPL) 

This is a free software, copyleft license. It is a modified version of the ordinary GNU GPL version 3. It has one added requirement: if you run the program on a server and let other users communicate with it there, your server must also allow them to download the source code corresponding to the program that it's running. If what's running there is your modified version of the program, the server's users must get the source code as you modified it.

\item Apache License

This is a permissive non-copyleft free software license. It has a few requirements that render it incompatible with the GNU GPL, such as strong prohibitions on the use of Apache-related names.
\end{itemize}
\subsection{License for Documentation and Creative Works}
\begin{itemize}
\item GNU Free Documentation License

This is a license intended for use on copylefted free documentation. The GNU FDL is recommended for textbooks and teaching materials for all topics. (``Documentation'' simply means textbooks and other teaching materials for using equipment or software.) It is also recommended to use the GNU FDL for dictionaries, encyclopedias, and any other works that provide information for practical use.
\item Creative Commons Attribution license (a.k.a. CC-BY)

This is a \emph{non-copyleft} free license that is good for art and entertainment works, and educational works. It is not used for software or documentation.

\item Creative Commons Attribution-Sharealike license (a.k.a. CC-BY-SA)

This is a \emph{copyleft} free license that is good for artistic and entertainment works, and educational works. It is not used for software or documentation.
\end{itemize}
